\documentclass[conference]{IEEEtran}
\IEEEoverridecommandlockouts
% The preceding line is only needed to identify funding in the first footnote. If that is unneeded, please comment it out.
%\usepackage{cite}
\usepackage{amsmath,amssymb,amsfonts}
% \usepackage{algorithmic}
\usepackage{algorithm}
\usepackage{algpseudocode}
\usepackage{graphicx}
\usepackage{textcomp}
\usepackage{xcolor}
\usepackage{biblatex} %Imports biblatex package
\addbibresource{citations.bib} %Import the bibliography file
\usepackage{subfiles} % Best loaded last in the preamble


\def\BibTeX{{\rm B\kern-.05em{\sc i\kern-.025em b}\kern-.08em
    T\kern-.1667em\lower.7ex\hbox{E}\kern-.125emX}}
\begin{document}

\title{Introduction to Radar Signal Processing Course Project}

\author{\IEEEauthorblockN{Doron Serebro} 
\IEEEauthorblockA{\textit{Dept. of Electrical Engineering} \\
\textit{Ben-Gurion University}\\
dorosner@post.bgu.ac.il}
}

\maketitle

\begin{abstract}
Multipath induced ghost targets are a major pain point for automotive radar. This work extends the multipath mitigation method suggested by Longman et. al. \cite{longman_multipath_2021} beyond vehicles to vulnerable road users such as pedestrians and cyclists. This is achieved by aggregating the radar detections to enable a more dense set of points per track. The method is evaluated on a dataset comprised of multiple real-world automotive scenarios.
\end{abstract}

\begin{IEEEkeywords}
Automotive radar, Multipath, Multipath mitigation
\end{IEEEkeywords}

\section{Introduction}
\subfile{sections/1_introduction}
 

\section{Problem Definition}
\subfile{sections/2_problem_definition}

\section{Multipath Mitigation Method}
\subfile{sections/3_method}

\section{Results}
\subfile{sections/4_results}

\section{Future Work}
The scope of this work was limited since it is a course project. Future work may improve on this by:
\begin{itemize}
    \item Increasing the number of estimated reflection points to the size of the larger cluster by performing a better point assosiation between the 2 candidate clusters (possible solution - center normalize both clusters and use KDTree for assosiation).
    \item suggest an evaluation metric that will express the agreement between the estimated reflection surface and the static radar detections (possible solution - count number of static points with a distance smaller than a given threshold to the estimated line segment).
\end{itemize}

\printbibliography %Prints bibliography
\vspace{12pt}

\end{document}
